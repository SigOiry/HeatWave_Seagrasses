% Options for packages loaded elsewhere
\PassOptionsToPackage{unicode}{hyperref}
\PassOptionsToPackage{hyphens}{url}
\PassOptionsToPackage{dvipsnames,svgnames,x11names}{xcolor}
%
\documentclass[
  number]{elsarticle}

\usepackage{amsmath,amssymb}
\usepackage{iftex}
\ifPDFTeX
  \usepackage[T1]{fontenc}
  \usepackage[utf8]{inputenc}
  \usepackage{textcomp} % provide euro and other symbols
\else % if luatex or xetex
  \usepackage{unicode-math}
  \defaultfontfeatures{Scale=MatchLowercase}
  \defaultfontfeatures[\rmfamily]{Ligatures=TeX,Scale=1}
\fi
\usepackage{lmodern}
\ifPDFTeX\else  
    % xetex/luatex font selection
\fi
% Use upquote if available, for straight quotes in verbatim environments
\IfFileExists{upquote.sty}{\usepackage{upquote}}{}
\IfFileExists{microtype.sty}{% use microtype if available
  \usepackage[]{microtype}
  \UseMicrotypeSet[protrusion]{basicmath} % disable protrusion for tt fonts
}{}
\makeatletter
\@ifundefined{KOMAClassName}{% if non-KOMA class
  \IfFileExists{parskip.sty}{%
    \usepackage{parskip}
  }{% else
    \setlength{\parindent}{0pt}
    \setlength{\parskip}{6pt plus 2pt minus 1pt}}
}{% if KOMA class
  \KOMAoptions{parskip=half}}
\makeatother
\usepackage{xcolor}
\setlength{\emergencystretch}{3em} % prevent overfull lines
\setcounter{secnumdepth}{5}
% Make \paragraph and \subparagraph free-standing
\makeatletter
\ifx\paragraph\undefined\else
  \let\oldparagraph\paragraph
  \renewcommand{\paragraph}{
    \@ifstar
      \xxxParagraphStar
      \xxxParagraphNoStar
  }
  \newcommand{\xxxParagraphStar}[1]{\oldparagraph*{#1}\mbox{}}
  \newcommand{\xxxParagraphNoStar}[1]{\oldparagraph{#1}\mbox{}}
\fi
\ifx\subparagraph\undefined\else
  \let\oldsubparagraph\subparagraph
  \renewcommand{\subparagraph}{
    \@ifstar
      \xxxSubParagraphStar
      \xxxSubParagraphNoStar
  }
  \newcommand{\xxxSubParagraphStar}[1]{\oldsubparagraph*{#1}\mbox{}}
  \newcommand{\xxxSubParagraphNoStar}[1]{\oldsubparagraph{#1}\mbox{}}
\fi
\makeatother


\providecommand{\tightlist}{%
  \setlength{\itemsep}{0pt}\setlength{\parskip}{0pt}}\usepackage{longtable,booktabs,array}
\usepackage{calc} % for calculating minipage widths
% Correct order of tables after \paragraph or \subparagraph
\usepackage{etoolbox}
\makeatletter
\patchcmd\longtable{\par}{\if@noskipsec\mbox{}\fi\par}{}{}
\makeatother
% Allow footnotes in longtable head/foot
\IfFileExists{footnotehyper.sty}{\usepackage{footnotehyper}}{\usepackage{footnote}}
\makesavenoteenv{longtable}
\usepackage{graphicx}
\makeatletter
\def\maxwidth{\ifdim\Gin@nat@width>\linewidth\linewidth\else\Gin@nat@width\fi}
\def\maxheight{\ifdim\Gin@nat@height>\textheight\textheight\else\Gin@nat@height\fi}
\makeatother
% Scale images if necessary, so that they will not overflow the page
% margins by default, and it is still possible to overwrite the defaults
% using explicit options in \includegraphics[width, height, ...]{}
\setkeys{Gin}{width=\maxwidth,height=\maxheight,keepaspectratio}
% Set default figure placement to htbp
\makeatletter
\def\fps@figure{htbp}
\makeatother

\makeatletter
\@ifpackageloaded{caption}{}{\usepackage{caption}}
\AtBeginDocument{%
\ifdefined\contentsname
  \renewcommand*\contentsname{Table of contents}
\else
  \newcommand\contentsname{Table of contents}
\fi
\ifdefined\listfigurename
  \renewcommand*\listfigurename{List of Figures}
\else
  \newcommand\listfigurename{List of Figures}
\fi
\ifdefined\listtablename
  \renewcommand*\listtablename{List of Tables}
\else
  \newcommand\listtablename{List of Tables}
\fi
\ifdefined\figurename
  \renewcommand*\figurename{Figure}
\else
  \newcommand\figurename{Figure}
\fi
\ifdefined\tablename
  \renewcommand*\tablename{Table}
\else
  \newcommand\tablename{Table}
\fi
}
\@ifpackageloaded{float}{}{\usepackage{float}}
\floatstyle{ruled}
\@ifundefined{c@chapter}{\newfloat{codelisting}{h}{lop}}{\newfloat{codelisting}{h}{lop}[chapter]}
\floatname{codelisting}{Listing}
\newcommand*\listoflistings{\listof{codelisting}{List of Listings}}
\makeatother
\makeatletter
\makeatother
\makeatletter
\@ifpackageloaded{caption}{}{\usepackage{caption}}
\@ifpackageloaded{subcaption}{}{\usepackage{subcaption}}
\makeatother
\ifLuaTeX
  \usepackage{selnolig}  % disable illegal ligatures
\fi
\usepackage[]{natbib}
\bibliographystyle{elsarticle-num}
\usepackage{bookmark}

\IfFileExists{xurl.sty}{\usepackage{xurl}}{} % add URL line breaks if available
\urlstyle{same} % disable monospaced font for URLs
\hypersetup{
  pdftitle={Draft -- Effect of Atmospheric Heatwaves on Reflectance and Pigment Composition of Intertidal Nanozostera noltei -- Draft},
  pdfauthor={Simon Oiry; Bede Ffinian Rowe Davies; Philippe Rosa; Pierre Gernez; Laurent Barillé},
  pdfkeywords={Remote Sensing, Pigment Composition, Seagrass, Coastal
Ecosystems, Heatwaves},
  colorlinks=true,
  linkcolor={blue},
  filecolor={Maroon},
  citecolor={Blue},
  urlcolor={Blue},
  pdfcreator={LaTeX via pandoc}}

\setlength{\parindent}{6pt}
\begin{document}

\begin{frontmatter}
\title{Draft -- Effect of Atmospheric Heatwaves on Reflectance and
Pigment Composition of Intertidal \emph{Nanozostera noltei} -- Draft}
\author[1]{Simon Oiry%
\corref{cor1}%
}
 \ead{oirysimon@gmail.com} 
\author[1]{Bede Ffinian Rowe Davies%
%
}

\author[1]{Philippe Rosa%
%
}

\author[1]{Pierre Gernez%
%
}

\author[1]{Laurent Barillé%
%
}


\affiliation[1]{organization={Institut des Substances et Organismes de
la Mer, ISOMer, Nantes Université, UR 2160, F-44000 Nantes,
France},,postcodesep={}}

\cortext[cor1]{Corresponding author}





        
\begin{abstract}
To be written
\end{abstract}





\begin{keyword}
    Remote Sensing \sep Pigment Composition \sep Seagrass \sep Coastal
Ecosystems \sep 
    Heatwaves
\end{keyword}
\end{frontmatter}
    
\section{Introduction}\label{introduction}

Intertidal seagrasses play a crucial role in the ecosystem by providing
habitats and feeding grounds for various marine species, supporting rich
marine biodiversity, and contributing significantly to primary
production and carbon sequestration
\citep{unsworth2022planetary, sousa2019blue}. These seagrasses are
essential in maintaining the health of coastal ecosystems by stabilizing
sediments, filtering water, and serving as indicators of environmental
changes due to their sensitivity to water quality variations
\citep{zoffoli2021decadal}. The interactions between seagrass meadows
and their associated herbivores further enhance the delivery of
ecosystem services, including coastal protection and fisheries support
\citep{jankowska2019stabilizing, zoffoli2023remote, gardner2018global}.
Understanding and preserving these ecosystems are vital for maintaining
the biodiversity and productivity of coastal regions
\citep{scott2018role, ramesh2020seagrass}.

Despite their crucial role in marine ecosystems, intertidal seagrasses
face numerous threats that compromise their health and functionality.
Coastal development and human activities are primary threats. These
activities not only reduce the available habitat for seagrasses but also
increase water turbidity, which limits light penetration and hampers
photosynthesis \citep{waycott2009accelerating}. Seagrasses are also
threatened by nutrient enrichment from agricultural and urban runoff,
which can lead to eutrophication. This condition promotes the overgrowth
of algal blooms that compete with seagrasses for light and nutrients,
further stressing these important plants \citep{thomsen2023meadow} (Oiry
et al.~2024). Pollution from industrial and municipal sources introduces
harmful chemicals and heavy metals into coastal waters, posing toxic
risks to seagrass health. These pollutants can affect the physiological
processes of seagrasses, reducing their growth and survival rates
\citep{sevgi2022bitkilerde} Additionally, invasive species can
outcompete native seagrasses for resources, altering community structure
and function \citep{simpson2016distribution}.

Heatwaves, exacerbated by climate change, represent a significant and
growing threat to seagrasses. The therms heatwave can refer to both
marine and atmospheric heatwave. \citep{hobday2016hierarchical} defines
Marine Heatwaves (MHW) as a a prolonged discrete anomalously warm water
event that can be described by its duration, intensity, rate of
evolution, and spatial extent. Specifically, an anomalously warm event
is considered a MHW if it lasts for five or more days, with temperatures
warmer than the 90th percentile based on a 30-year historical baseline
period. on the other hand, Atmospheric Heatwaves (AHW) has been defined
by \citep{perkins2013measurement} as a period of at least three
consecutive days with temperatures exceeding the 90th percentile for
that time of year. These extreme temperature events can cause severe
physiological stress, affecting growth, reproduction, and survival
\citep{sawall2021chronically, deguette2022physiological}. Heatwaves can
intensify the cumulative effects of other stressors such as overgrazing
and seed burial, leading to compromised sexual recruitment
\citep{guerrero2020heat}.


\renewcommand\refname{Bibliography}
  \bibliography{library.bib}


\end{document}
