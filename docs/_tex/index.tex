% Options for packages loaded elsewhere
\PassOptionsToPackage{unicode}{hyperref}
\PassOptionsToPackage{hyphens}{url}
\PassOptionsToPackage{dvipsnames,svgnames,x11names}{xcolor}
%
\documentclass[
  number]{elsarticle}

\usepackage{amsmath,amssymb}
\usepackage{iftex}
\ifPDFTeX
  \usepackage[T1]{fontenc}
  \usepackage[utf8]{inputenc}
  \usepackage{textcomp} % provide euro and other symbols
\else % if luatex or xetex
  \usepackage{unicode-math}
  \defaultfontfeatures{Scale=MatchLowercase}
  \defaultfontfeatures[\rmfamily]{Ligatures=TeX,Scale=1}
\fi
\usepackage{lmodern}
\ifPDFTeX\else  
    % xetex/luatex font selection
\fi
% Use upquote if available, for straight quotes in verbatim environments
\IfFileExists{upquote.sty}{\usepackage{upquote}}{}
\IfFileExists{microtype.sty}{% use microtype if available
  \usepackage[]{microtype}
  \UseMicrotypeSet[protrusion]{basicmath} % disable protrusion for tt fonts
}{}
\makeatletter
\@ifundefined{KOMAClassName}{% if non-KOMA class
  \IfFileExists{parskip.sty}{%
    \usepackage{parskip}
  }{% else
    \setlength{\parindent}{0pt}
    \setlength{\parskip}{6pt plus 2pt minus 1pt}}
}{% if KOMA class
  \KOMAoptions{parskip=half}}
\makeatother
\usepackage{xcolor}
\setlength{\emergencystretch}{3em} % prevent overfull lines
\setcounter{secnumdepth}{5}
% Make \paragraph and \subparagraph free-standing
\makeatletter
\ifx\paragraph\undefined\else
  \let\oldparagraph\paragraph
  \renewcommand{\paragraph}{
    \@ifstar
      \xxxParagraphStar
      \xxxParagraphNoStar
  }
  \newcommand{\xxxParagraphStar}[1]{\oldparagraph*{#1}\mbox{}}
  \newcommand{\xxxParagraphNoStar}[1]{\oldparagraph{#1}\mbox{}}
\fi
\ifx\subparagraph\undefined\else
  \let\oldsubparagraph\subparagraph
  \renewcommand{\subparagraph}{
    \@ifstar
      \xxxSubParagraphStar
      \xxxSubParagraphNoStar
  }
  \newcommand{\xxxSubParagraphStar}[1]{\oldsubparagraph*{#1}\mbox{}}
  \newcommand{\xxxSubParagraphNoStar}[1]{\oldsubparagraph{#1}\mbox{}}
\fi
\makeatother


\providecommand{\tightlist}{%
  \setlength{\itemsep}{0pt}\setlength{\parskip}{0pt}}\usepackage{longtable,booktabs,array}
\usepackage{calc} % for calculating minipage widths
% Correct order of tables after \paragraph or \subparagraph
\usepackage{etoolbox}
\makeatletter
\patchcmd\longtable{\par}{\if@noskipsec\mbox{}\fi\par}{}{}
\makeatother
% Allow footnotes in longtable head/foot
\IfFileExists{footnotehyper.sty}{\usepackage{footnotehyper}}{\usepackage{footnote}}
\makesavenoteenv{longtable}
\usepackage{graphicx}
\makeatletter
\def\maxwidth{\ifdim\Gin@nat@width>\linewidth\linewidth\else\Gin@nat@width\fi}
\def\maxheight{\ifdim\Gin@nat@height>\textheight\textheight\else\Gin@nat@height\fi}
\makeatother
% Scale images if necessary, so that they will not overflow the page
% margins by default, and it is still possible to overwrite the defaults
% using explicit options in \includegraphics[width, height, ...]{}
\setkeys{Gin}{width=\maxwidth,height=\maxheight,keepaspectratio}
% Set default figure placement to htbp
\makeatletter
\def\fps@figure{htbp}
\makeatother

\makeatletter
\@ifpackageloaded{caption}{}{\usepackage{caption}}
\AtBeginDocument{%
\ifdefined\contentsname
  \renewcommand*\contentsname{Table of contents}
\else
  \newcommand\contentsname{Table of contents}
\fi
\ifdefined\listfigurename
  \renewcommand*\listfigurename{List of Figures}
\else
  \newcommand\listfigurename{List of Figures}
\fi
\ifdefined\listtablename
  \renewcommand*\listtablename{List of Tables}
\else
  \newcommand\listtablename{List of Tables}
\fi
\ifdefined\figurename
  \renewcommand*\figurename{Figure}
\else
  \newcommand\figurename{Figure}
\fi
\ifdefined\tablename
  \renewcommand*\tablename{Table}
\else
  \newcommand\tablename{Table}
\fi
}
\@ifpackageloaded{float}{}{\usepackage{float}}
\floatstyle{ruled}
\@ifundefined{c@chapter}{\newfloat{codelisting}{h}{lop}}{\newfloat{codelisting}{h}{lop}[chapter]}
\floatname{codelisting}{Listing}
\newcommand*\listoflistings{\listof{codelisting}{List of Listings}}
\makeatother
\makeatletter
\makeatother
\makeatletter
\@ifpackageloaded{caption}{}{\usepackage{caption}}
\@ifpackageloaded{subcaption}{}{\usepackage{subcaption}}
\makeatother
\ifLuaTeX
  \usepackage{selnolig}  % disable illegal ligatures
\fi
\usepackage[]{natbib}
\bibliographystyle{elsarticle-num}
\usepackage{bookmark}

\IfFileExists{xurl.sty}{\usepackage{xurl}}{} % add URL line breaks if available
\urlstyle{same} % disable monospaced font for URLs
\hypersetup{
  pdftitle={Draft -- Discriminating Seagrasses From Green Macroalgae in European Intertidal areas using high resolution multispectral drone imagery -- Draft},
  pdfauthor={Simon Oiry; Bede Ffinian Rowe Davies; Ana I. Sousa; Philippe Rosa; Maria Laura Zoffoli; Guillaume Brunier; Pierre Gernez; Laurent Barillé},
  pdfkeywords={Drone, Remote Sensing, Seagrass, Coastal
Ecosystems, Neural Network},
  colorlinks=true,
  linkcolor={blue},
  filecolor={Maroon},
  citecolor={Blue},
  urlcolor={Blue},
  pdfcreator={LaTeX via pandoc}}

\setlength{\parindent}{6pt}
\begin{document}

\begin{frontmatter}
\title{Draft -- Discriminating Seagrasses From Green Macroalgae in
European Intertidal areas using high resolution multispectral drone
imagery -- Draft}
\author[1]{Simon Oiry%
\corref{cor1}%
}
 \ead{oirysimon@gmail.com} 
\author[1]{Bede Ffinian Rowe Davies%
%
}

\author[2]{Ana I. Sousa%
%
}

\author[1]{Philippe Rosa%
%
}

\author[3]{Maria Laura Zoffoli%
%
}

\author[4]{Guillaume Brunier%
%
}

\author[1]{Pierre Gernez%
%
}

\author[1]{Laurent Barillé%
%
}


\affiliation[1]{organization={Institut des Substances et Organismes de
la Mer, ISOMer, Nantes Université, UR 2160, F-44000 Nantes,
France},,postcodesep={}}
\affiliation[2]{organization={ECOMARE - Laboratory for Innovation and
Sustainability of Marine Biological Resources, CESAM -- Centre for
Environmental and Marine Studies, Department of Biology, University of
Aveiro, Campus Universitário de Santiago, 3810-193 Aveiro,
Portugal},,postcodesep={}}
\affiliation[3]{organization={Consiglio Nazionale delle Ricerche,
Istituto di Scienze Marine (CNR-ISMAR), 00133 Rome,
Italy},,postcodesep={}}
\affiliation[4]{organization={BRGM French Geological Survey, Cayenne
97300, French Guiana},,postcodesep={}}

\cortext[cor1]{Corresponding author}








        
\begin{abstract}
Coastal areas support seagrass meadows, which offer crucial ecosystem
services including erosion control and carbon sequestration. However,
these areas are increasingly impacted by human activities, leading to
habitat fragmentation and seagrass decline. In situ surveys,
traditionally performed to monitor these ecosystems face limitations on
temporal and spatial coverage, particularly in intertidal zones,
prompting the addition of satellite data within monitoring programs.
Yet, satellite remote sensing struggles with spatial and spectral
resolution, making it difficult to discriminate seagrass from other
macrophytes in highly heterogenous meadows. Drone (unmanned aerial
vehicles -- UAV) images at a very high spatial resolution offer a
promising solution to address challenges related to spatial
heterogeneity and intrapixel mixture. This study focuses on using drone
acquisitions with a ten spectral band sensor mirroring those of
Sentinel-2, for mapping intertidal macrophytes and effectively
discriminating between seagrass and green macroalgae. Nine drone flights
were conducted at two different altitudes (12 m and 120 m) across
heterogeneous intertidal European habitats in France and Portugal. Low
altitude flights were used to train a Deep Learning classifier based on
Neural Networks to discrimintate among five taxonomic classes of
intertidal vegetation: Magnoliopsida (Seagrass), Chlorophyceae (Green
macroalgae), Phaeophyceae (Brown algae), Rhodophyceae (Red macroalgae)
and benthic Bacillariophyceae (Diatoms). Classification of drone imagery
resulted in an overall accuracy of 94\% across all the sites and images,
covering a total area of 467 000 m². The model exhibited an accuracy of
96.4\% in identifying seagrass. Importantly, seagrass and green algae
can be discriminated, although they share the same pigment composition.
As the algorithm was developed for a multispectral camera with ten
spectral bands in the visible and near-infrared, it could be adapted to
the Multi-Spectral Instrument (MSI) onboard Sentinel-2 thus offering
promising perspectives for satellite remote sensing of intertidal
biodiversity over lager scales.
\end{abstract}





\begin{keyword}
    Drone \sep Remote Sensing \sep Seagrass \sep Coastal
Ecosystems \sep 
    Neural Network
\end{keyword}
\end{frontmatter}
    



  \bibliography{library.bib}


\end{document}
