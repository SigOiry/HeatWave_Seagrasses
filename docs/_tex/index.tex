% Options for packages loaded elsewhere
\PassOptionsToPackage{unicode}{hyperref}
\PassOptionsToPackage{hyphens}{url}
\PassOptionsToPackage{dvipsnames,svgnames,x11names}{xcolor}
%
\documentclass[
  number]{elsarticle}

\usepackage{amsmath,amssymb}
\usepackage{iftex}
\ifPDFTeX
  \usepackage[T1]{fontenc}
  \usepackage[utf8]{inputenc}
  \usepackage{textcomp} % provide euro and other symbols
\else % if luatex or xetex
  \usepackage{unicode-math}
  \defaultfontfeatures{Scale=MatchLowercase}
  \defaultfontfeatures[\rmfamily]{Ligatures=TeX,Scale=1}
\fi
\usepackage{lmodern}
\ifPDFTeX\else  
    % xetex/luatex font selection
\fi
% Use upquote if available, for straight quotes in verbatim environments
\IfFileExists{upquote.sty}{\usepackage{upquote}}{}
\IfFileExists{microtype.sty}{% use microtype if available
  \usepackage[]{microtype}
  \UseMicrotypeSet[protrusion]{basicmath} % disable protrusion for tt fonts
}{}
\makeatletter
\@ifundefined{KOMAClassName}{% if non-KOMA class
  \IfFileExists{parskip.sty}{%
    \usepackage{parskip}
  }{% else
    \setlength{\parindent}{0pt}
    \setlength{\parskip}{6pt plus 2pt minus 1pt}}
}{% if KOMA class
  \KOMAoptions{parskip=half}}
\makeatother
\usepackage{xcolor}
\setlength{\emergencystretch}{3em} % prevent overfull lines
\setcounter{secnumdepth}{5}
% Make \paragraph and \subparagraph free-standing
\makeatletter
\ifx\paragraph\undefined\else
  \let\oldparagraph\paragraph
  \renewcommand{\paragraph}{
    \@ifstar
      \xxxParagraphStar
      \xxxParagraphNoStar
  }
  \newcommand{\xxxParagraphStar}[1]{\oldparagraph*{#1}\mbox{}}
  \newcommand{\xxxParagraphNoStar}[1]{\oldparagraph{#1}\mbox{}}
\fi
\ifx\subparagraph\undefined\else
  \let\oldsubparagraph\subparagraph
  \renewcommand{\subparagraph}{
    \@ifstar
      \xxxSubParagraphStar
      \xxxSubParagraphNoStar
  }
  \newcommand{\xxxSubParagraphStar}[1]{\oldsubparagraph*{#1}\mbox{}}
  \newcommand{\xxxSubParagraphNoStar}[1]{\oldsubparagraph{#1}\mbox{}}
\fi
\makeatother


\providecommand{\tightlist}{%
  \setlength{\itemsep}{0pt}\setlength{\parskip}{0pt}}\usepackage{longtable,booktabs,array}
\usepackage{calc} % for calculating minipage widths
% Correct order of tables after \paragraph or \subparagraph
\usepackage{etoolbox}
\makeatletter
\patchcmd\longtable{\par}{\if@noskipsec\mbox{}\fi\par}{}{}
\makeatother
% Allow footnotes in longtable head/foot
\IfFileExists{footnotehyper.sty}{\usepackage{footnotehyper}}{\usepackage{footnote}}
\makesavenoteenv{longtable}
\usepackage{graphicx}
\makeatletter
\def\maxwidth{\ifdim\Gin@nat@width>\linewidth\linewidth\else\Gin@nat@width\fi}
\def\maxheight{\ifdim\Gin@nat@height>\textheight\textheight\else\Gin@nat@height\fi}
\makeatother
% Scale images if necessary, so that they will not overflow the page
% margins by default, and it is still possible to overwrite the defaults
% using explicit options in \includegraphics[width, height, ...]{}
\setkeys{Gin}{width=\maxwidth,height=\maxheight,keepaspectratio}
% Set default figure placement to htbp
\makeatletter
\def\fps@figure{htbp}
\makeatother

\makeatletter
\@ifpackageloaded{caption}{}{\usepackage{caption}}
\AtBeginDocument{%
\ifdefined\contentsname
  \renewcommand*\contentsname{Table of contents}
\else
  \newcommand\contentsname{Table of contents}
\fi
\ifdefined\listfigurename
  \renewcommand*\listfigurename{List of Figures}
\else
  \newcommand\listfigurename{List of Figures}
\fi
\ifdefined\listtablename
  \renewcommand*\listtablename{List of Tables}
\else
  \newcommand\listtablename{List of Tables}
\fi
\ifdefined\figurename
  \renewcommand*\figurename{Figure}
\else
  \newcommand\figurename{Figure}
\fi
\ifdefined\tablename
  \renewcommand*\tablename{Table}
\else
  \newcommand\tablename{Table}
\fi
}
\@ifpackageloaded{float}{}{\usepackage{float}}
\floatstyle{ruled}
\@ifundefined{c@chapter}{\newfloat{codelisting}{h}{lop}}{\newfloat{codelisting}{h}{lop}[chapter]}
\floatname{codelisting}{Listing}
\newcommand*\listoflistings{\listof{codelisting}{List of Listings}}
\makeatother
\makeatletter
\makeatother
\makeatletter
\@ifpackageloaded{caption}{}{\usepackage{caption}}
\@ifpackageloaded{subcaption}{}{\usepackage{subcaption}}
\makeatother
\ifLuaTeX
  \usepackage{selnolig}  % disable illegal ligatures
\fi
\usepackage[]{natbib}
\bibliographystyle{elsarticle-num}
\usepackage{bookmark}

\IfFileExists{xurl.sty}{\usepackage{xurl}}{} % add URL line breaks if available
\urlstyle{same} % disable monospaced font for URLs
\hypersetup{
  pdftitle={Draft -- Effect of Atmospheric Heatwaves on Reflectance and Pigment Composition of Intertidal Nanozostera noltei -- Draft},
  pdfauthor={Simon Oiry; Bede Ffinian Rowe Davies; Philippe Rosa; Augustin Debly; Pierre Gernez; Laurent Barillé},
  pdfkeywords={Remote Sensing, Pigment Composition, Seagrass, Coastal
Ecosystems, Heatwaves},
  colorlinks=true,
  linkcolor={blue},
  filecolor={Maroon},
  citecolor={Blue},
  urlcolor={Blue},
  pdfcreator={LaTeX via pandoc}}

\setlength{\parindent}{6pt}
\begin{document}

\begin{frontmatter}
\title{Draft -- Effect of Atmospheric Heatwaves on Reflectance and
Pigment Composition of Intertidal \emph{Nanozostera noltei} -- Draft}
\author[1]{Simon Oiry%
\corref{cor1}%
}
 \ead{oirysimon@gmail.com} 
\author[1]{Bede Ffinian Rowe Davies%
%
}

\author[1]{Philippe Rosa%
%
}

\author[1]{Augustin Debly%
%
}

\author[1]{Pierre Gernez%
%
}

\author[1]{Laurent Barillé%
%
}


\affiliation[1]{organization={Institut des Substances et Organismes de
la Mer, ISOMer, Nantes Université, UR 2160, F-44000 Nantes,
France},,postcodesep={}}

\cortext[cor1]{Corresponding author}






        
\begin{abstract}
To be written
\end{abstract}





\begin{keyword}
    Remote Sensing \sep Pigment Composition \sep Seagrass \sep Coastal
Ecosystems \sep 
    Heatwaves
\end{keyword}
\end{frontmatter}
    
\section{Introduction}\label{introduction}

Intertidal seagrasses play a crucial role in the ecosystem by providing
habitats and feeding grounds for various marine species, supporting rich
marine biodiversity, and contributing significantly to primary
production and carbon sequestration
\citep{unsworth2022planetary, sousa2019blue}. These seagrasses are
essential in maintaining the health of coastal ecosystems by stabilizing
sediments, filtering water, and serving as indicators of environmental
changes due to their sensitivity to water quality variations
\citep{zoffoli2021decadal}. The interactions between seagrass meadows
and their associated herbivores further enhance the delivery of
ecosystem services, including coastal protection and fisheries support
\citep{jankowska2019stabilizing, zoffoli2023remote, gardner2018global}.
Understanding and preserving these ecosystems are vital for maintaining
the biodiversity and productivity of coastal regions
\citep{scott2018role, ramesh2020seagrass}.

Despite their crucial role in marine ecosystems, intertidal seagrasses
face numerous threats that compromise their health and functionality.
Coastal development and human activities are primary threats. These
activities not only reduce the available habitat for seagrasses but also
increase water turbidity, which limits light penetration and hampers
photosynthesis \citep{waycott2009accelerating}. Seagrasses are also
threatened by nutrient enrichment from agricultural and urban runoff,
which can lead to eutrophication. This condition promotes the overgrowth
of algal blooms that compete with seagrasses for light and nutrients,
further stressing these important plants \citep{thomsen2023meadow} (Oiry
et al.~2024). Pollution from industrial and agricultural fields sources
introduces harmful chemicals and heavy metals into coastal waters,
posing toxic risks to seagrass health. These pollutants can affect the
physiological processes of seagrasses, reducing their growth and
survival rates \citep{sevgi2022bitkilerde} Additionally, invasive
species can out compete native seagrasses for resources, altering
community structure and function \citep{simpson2016distribution}.

Heatwaves, exacerbated by climate change, pose a growing threat to
seagrasses. Marine Heatwaves (MHW), defined by
\citep{hobday2016hierarchical} as prolonged discrete anomalously warm
water events, and Atmospheric Heatwaves (AHW), defined by
\citep{perkins2013measurement} as periods of at least three consecutive
days with temperatures exceeding the 90th percentile, cause severe
physiological stress on seagrasses
\citep{sawall2021chronically, deguette2022physiological}. At the
interface between the land and oceans, intertidal seagrasses are exposed
to both MHW and AHW. Heatwaves have profound impacts on seagrasses, with
their effects varying based on species and geographic location. For
instance, the seagrass \emph{Zostera marina} exhibits high
susceptibility to elevated sea surface temperatures during winter and
spring, leading to advanced flowering, high mortality rates, and reduced
biomass \citep{sawall2021chronically}. Similarly, \emph{Cymodocea
nodosa} shows increased photosynthetic activity during heatwaves but
suffers negative effects on photosynthetic performance and leaf biomass
during recovery \citep{deguette2022physiological}. Additionally,
different populations of \emph{Zostera marina} along the European
thermal gradient exhibit varied photophysiological responses during the
recovery phase of heatwaves, indicating differential adaptation
capabilities among populations \citep{winters2011effects}. These events
intensify other stressors, such as overgrazing and seed burial,
compromising sexual recruitment \citep{guerrero2020heat}.

Bleaching and browning events of seagrass beds have been observed
following episodes of intense heat along the Brittany coast of France
(Pers. obs.) then affecting leaf color, which are expected to alter leaf
reflectance. Remote sensing is increasingly being utilized to monitor
marine ecosystems, including seagrass meadows. By using spectral
indices, such as the Normalized Difference Vegetation Index (NDVI) and
the Soil-Adjusted Vegetation Index (SAVI), or by analyzing specific
spectral patterns, remote sensing can effectively quantify vegetation
health over time
\citep{huete2012vegetation, kloos2021agricultural, carlan2020identifying, akbar2020mangrove}.
Through the Water Framework Directive and the Marine Strategy Framework
Directive, Europe is promoting remote sensing techniques for habitat
mapping, as these methods enable the monitoring of extensive areas over
time \citep{papathanasopoulou2019satellite}. This study will
experimentally test the hypothesis that warm events modify the pigment
composition and reflectance of seagrass, linking these changes with
satellite remote sensing.

\section{Material \& Methods}\label{material-methods}

\subsection{Sampling and Acclimation of
seagrasses}\label{sampling-and-acclimation-of-seagrasses}

Seagrass was sampled from a \emph{Nanozostera noltei} (dwarf eelgrass,
syn. Zostera noltei) meadow on Noirmoutier Island, France (46°57'32.0''N
2°10'37.0''W) at low tide in June 2024. A shovel was used to sample
seagrass from an area of 30 cm by 15 cm and 10 cm deep, maintaining the
sediment structure and avoiding damage to the rhizomes and the leafs of
the seagrass. The seagrass, along with sediment, meiofauna, and
macrofauna, was placed in plastic trays. To avoid hydric stress during
transportation, seawater was added to each tray. Simultaneously,
seawater was sampled from a nearby site and transported to the lab,
where it was filtered using a 0.22 µm nitrocellulose filter to remove
all suspended particulate matter. This water was used in the acclimation
tank and the intertidal chambers. The seagrasses were acclimated for two
weeks with a water temperature of 17°C, matching the temperature at the
time of sampling, and with light of 150 µmol.s-1.m-2 of PAR photons
\citep{akbar2020mangrove}. A wave generator was used in the tank to
circulate and reoxygenate the water.


\renewcommand\refname{Bibliography}
  \bibliography{library.bib}


\end{document}
